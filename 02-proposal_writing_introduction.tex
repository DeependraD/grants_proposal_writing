\PassOptionsToPackage{unicode=true}{hyperref} % options for packages loaded elsewhere
\PassOptionsToPackage{hyphens}{url}
\PassOptionsToPackage{dvipsnames,svgnames*,x11names*}{xcolor}
%
\documentclass[ignorenonframetext,aspectratio=169]{beamer}
\usepackage{pgfpages}
\setbeamertemplate{caption}[numbered]
\setbeamertemplate{caption label separator}{: }
\setbeamercolor{caption name}{fg=normal text.fg}
\beamertemplatenavigationsymbolsempty
% Prevent slide breaks in the middle of a paragraph:
\widowpenalties 1 10000
\raggedbottom
\setbeamertemplate{part page}{
\centering
\begin{beamercolorbox}[sep=16pt,center]{part title}
  \usebeamerfont{part title}\insertpart\par
\end{beamercolorbox}
}
\setbeamertemplate{section page}{
\centering
\begin{beamercolorbox}[sep=12pt,center]{part title}
  \usebeamerfont{section title}\insertsection\par
\end{beamercolorbox}
}
\setbeamertemplate{subsection page}{
\centering
\begin{beamercolorbox}[sep=8pt,center]{part title}
  \usebeamerfont{subsection title}\insertsubsection\par
\end{beamercolorbox}
}
\AtBeginPart{
  \frame{\partpage}
}
\AtBeginSection{
  \ifbibliography
  \else
    \frame{\sectionpage}
  \fi
}
\AtBeginSubsection{
  \frame{\subsectionpage}
}
\usepackage{lmodern}
\usepackage{amssymb,amsmath}
\usepackage{ifxetex,ifluatex}
\usepackage{fixltx2e} % provides \textsubscript
\ifnum 0\ifxetex 1\fi\ifluatex 1\fi=0 % if pdftex
  \usepackage[T1]{fontenc}
  \usepackage[utf8]{inputenc}
  \usepackage{textcomp} % provides euro and other symbols
\else % if luatex or xelatex
  \usepackage{unicode-math}
  \defaultfontfeatures{Ligatures=TeX,Scale=MatchLowercase}
\fi
\usetheme[]{Frankfurt}
\usecolortheme{beaver}
\usefonttheme{structuresmallcapsserif}
% use upquote if available, for straight quotes in verbatim environments
\IfFileExists{upquote.sty}{\usepackage{upquote}}{}
% use microtype if available
\IfFileExists{microtype.sty}{%
\usepackage[]{microtype}
\UseMicrotypeSet[protrusion]{basicmath} % disable protrusion for tt fonts
}{}
\IfFileExists{parskip.sty}{%
\usepackage{parskip}
}{% else
\setlength{\parindent}{0pt}
\setlength{\parskip}{6pt plus 2pt minus 1pt}
}
\usepackage{xcolor}
\usepackage{hyperref}
\hypersetup{
            pdftitle={Proposal writing: Introduction},
            pdfauthor={Deependra Dhakal},
            colorlinks=true,
            linkcolor=red,
            filecolor=Maroon,
            citecolor=blue,
            urlcolor=red,
            breaklinks=true}
\urlstyle{same}  % don't use monospace font for urls
\newif\ifbibliography
\setlength{\emergencystretch}{3em}  % prevent overfull lines
\providecommand{\tightlist}{%
  \setlength{\itemsep}{0pt}\setlength{\parskip}{0pt}}
\setcounter{secnumdepth}{0}

% set default figure placement to htbp
\makeatletter
\def\fps@figure{htbp}
\makeatother

% define new footline defination
\setbeamertemplate{footline}
{
    \leavevmode%
    \hbox{%
        \begin{beamercolorbox}[wd = .5\paperwidth, ht = 1ex, dp = 1ex, center]{author in head/foot}%
            Text in footer
        \end{beamercolorbox}%
        \begin{beamercolorbox}[wd = .5\paperwidth, ht = 1ex, dp = 1ex, center]{date in head/foot}%
            \insertframenumber{}
        \end{beamercolorbox}
    }%
    \vskip3pt%
}
% add footline defination to the beamer template
\addtobeamertemplate{footline}{\begin{center}\rule{0.6\paperwidth}{0.4pt}\end{center}\vspace*{-1ex}}{}

% % set background in a TikZ node for modifications
\usepackage{tikz}
\usepackage[absolute,overlay]{textpos}
\setbeamertemplate{title page}{
\tikz\node[opacity=0.5] {\includegraphics[width=0.8\paperwidth]{the_proposal.jpg}};
% \begin{textblock}{7.5}(1,2.8)\usebeamerfont{title}
\begin{textblock}{15}(1.5,2.8)\usebeamerfont{title} % 1.5 is x position in page
{\color{white}\raggedright\par\inserttitle}
\end{textblock}
\begin{textblock}{7.5}(1.5,7) % 1.5 is x position in page
{\color{white}\raggedright{\insertauthor}\mbox{}\\[0.2cm]
\insertdate}
\end{textblock}} % dd_rookie modified, figure is top aligned

% % set caption font size
% % note that beamer presentation native captions have their own configs
% \usepackage{caption}
% \captionsetup{font=footnotesize}

% this font option is amenable for beamer
\setbeamerfont{caption}{size=\tiny}

% some beamer themes naturally might not support navigation symbols
% \setbeamertemplate{navigation symbols}{} % remove navigation symbols

\setbeamertemplate{footline}[page number] % insert page number in footline

% \setbeamertemplate{navigation symbols}{slide} % insert slide indication in navigation
% \setbeamertemplate{navigation symbols}{frame} % insert frame indication in navigation
% \setbeamertemplate{navigation symbols}{section} % insert section indication in navigation
% \setbeamertemplate{navigation symbols}{subsection} % insert subsection indication in navigation

% \AtBeginSubsection{} % supress subsection display

\title{Proposal writing: Introduction}
\author{Deependra Dhakal}
\providecommand{\institute}[1]{}
\institute{GAASC, Baitadi \and Tribhuwan University}
\date{Academic year 2019-2020}

\begin{document}
\frame{\titlepage}

\begin{frame}
\tableofcontents[hideallsubsections]
\end{frame}
\hypertarget{basics}{%
\section{Basics}\label{basics}}

\hypertarget{organization}{%
\section{Organization}\label{organization}}

\begin{frame}{Background}
\protect\hypertarget{background}{}

\begin{itemize}
\tightlist
\item
  the subject domain
\item
  connecting two or more domains, if present
\item
  is the subject of study extensively researched ?
\item
  if researched already what problems have been answered ?
\item
  what remains to be answered ?
\item
  is there an exception to the problem and solution posed for it earlier
  ? you could address that.
\item
  what are the popular suggestions for the problem, if any ?
\end{itemize}

\end{frame}

\begin{frame}{Problem}
\protect\hypertarget{problem}{}

\begin{itemize}
\tightlist
\item
  how the problem is a problem ?

  \begin{itemize}
  \tightlist
  \item
    human interest
  \item
    economic welfare
  \item
    ecosystem benefit
  \end{itemize}
\item
  does problem require special conditions to explore
\item
  why designing a research study for the problem now is right thing to
  do?

  \begin{itemize}
  \tightlist
  \item
    why not delay it ?
  \item
    how is it an unique problem ?
  \end{itemize}
\item
  write from the perspective of sponsor/reader not yourself
\end{itemize}

\end{frame}

\begin{frame}{What objectives to meet}
\protect\hypertarget{what-objectives-to-meet}{}

\begin{itemize}
\tightlist
\item
  be very specific
\item
  write only that which is verifiable/measurable
\item
  indicate immediacy;time frame during which problem will be addressed
\item
  practicable
\item
  essentially break down the topic
\item
  objectives are linearly related to each other; logical order
\end{itemize}

\end{frame}

\begin{frame}{How to meet those objectives}
\protect\hypertarget{how-to-meet-those-objectives}{}

\begin{itemize}
\tightlist
\item
  hypothesize
\item
  experimental runs
\item
  recording measurement/data
\item
  how are you better equipped than others to conduct a research
\end{itemize}

\end{frame}

\begin{frame}{Who will benefit}
\protect\hypertarget{who-will-benefit}{}

\end{frame}

\begin{frame}{Methods}
\protect\hypertarget{methods}{}

\begin{itemize}
\tightlist
\item
  Explain why you chose one methodological approach and not another.
\item
  Describe the major activities for reaching each objective.
\item
  Indicate the key project personnel/group who will carry out each
  activity.
\item
  Show the interrelationship among project activities.
\item
  Indicate how it will be feasible with given resources.
\item
  What will you observed
\item
  What will you analyse
\item
  Whay will you infer about
\item
  How will you collect data
\item
  How will you analyse data
\item
  How will you make inference
\end{itemize}

\end{frame}

\begin{frame}{Evaluation}
\protect\hypertarget{evaluation}{}

\begin{itemize}
\tightlist
\item
  Impact evaluation

  \begin{itemize}
  \tightlist
  \item
    Overall value
  \item
    Sustainability
  \item
    Replicability
  \end{itemize}
\item
  Who can evaluate
\item
  How long does the impact hold
\end{itemize}

\end{frame}

\begin{frame}{Budget and timeline}
\protect\hypertarget{budget-and-timeline}{}

\end{frame}

\begin{frame}{References}
\protect\hypertarget{references}{}

\end{frame}

\begin{frame}{Abstract/summary}
\protect\hypertarget{abstractsummary}{}

\end{frame}

\end{document}
